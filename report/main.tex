\documentclass[a4paper,11pt]{article}
\usepackage[utf8]{inputenc}
%\usepackage{fullpage}
\usepackage[frenchb]{babel}
\usepackage[T1]{fontenc}
\usepackage{graphicx}
\usepackage{ifpdf}
\usepackage{hyperref}
\usepackage{amsmath}

\title{\textsc{X6I0030} Recherche Opérationnelle \\ Mission planning pour une flotte de robots d’exploration}
\author{Théo Delalande-Delarbre \and Benjamin Sientzoff}
\date{\today}
\ifpdf
\hypersetup{
    pdfauthor={Théo Delalande-Delarbre, Benjamin Sientzoff},
    pdftitle={X6I0030 - Recherche Opérationnelle - Mission planning pour une flotte de robots d’exploration}
}
\fi

\begin{document}

\maketitle

% sommaire sur une nouvelle page
\newpage
\tableofcontents

\paragraph{}{
	Paragraphe d'introduction
}

\section{Preuve} % trouver un meilleur titre

\paragraph{Affirmation}{
	Nous avons donc ici obtenu une solution admissible et optimale pour cette instance du problème de voyageur de commerce en ne considérant que deux contraintes de l’ensemble (3 ′ ).
}

\begin{equation}
	\sum_{i,j= \in S} x_{ij} \leq |S| \less 1 , avec \forall S 2 \leq |S| \leq n \less 1
\end{equation}


\newpage

\paragraph{}{
	Paragraphe de conclusion
}


\end{document}

– La preuve de l’affirmation qui apparaît deux fois en gras dans le texte. Plus précisément, en résolvant un
problème ne contenant qu’un sous-ensemble des contraintes (3’), si on obtient une solution composée d’un
seul cycle alors cette solution est optimale.
– La correspondance entre les (double-)indices des variables du problème, et les indices utilisés dans GLPK,
– La récupération de la solution optimale retournée par GLPK et sa traduction en permutations, puis en produit
de cycles disjoints,
– Une analyse expérimentale effectuée à partir des instances fournies (temps CPU, nombre de contraintes
ajoutées pour résoudre le problème...). On considérera séparément les instances de la catégorie plat et
relief. Lesquelles sont les plus difficiles à résoudre avec l’algorithme proposé, pourquoi ?
– Éventuellement, des améliorations pourront être proposées suite à cette analyse expérimentale.
